\newpage%\新一页
\section{模型的假设}
针对本文题目,提出以下假设:
%1.	假设题目所给的数据真实可靠;
%注意:假设对整篇文章具有指导性,有时决定问题的难易。一定要注意假设的某种角度上的合理性,不能乱编,完全偏离事实或与题目要求相抵触。注意罗列要工整。
\begin{enumerate}
	\item 
	\item 
	\item 
	\item 
	\item
\end{enumerate}

\section{符号说明}
%这部分不要过页(删掉此句话)
\begin{center}%置中
	\begin{tabular}{cc}
		\toprule[1pt] 
		\makebox[0.15\textwidth][c]{符号} & \makebox[0.4\textwidth][c]{说明} \\  
		\hline
		$T_i$&小温区温度\\
		$T_i$&小温区温度\\    
		\bottomrule[1pt]
		%尽可能借鉴参考书上通常采用的符号,不宜自己乱定义符号,对于改进的一些模型,符号可以适当自己修正(下标、上标、参数等可以变,主符号最好与经典模型符号靠近)。
		%对文章自己创新的名词需要特别解释。其他符号要进行说明,注意罗列要工整。如“ ~第 种疗法的第 项指标值”等,注意格式统一,不要出现零乱或前后不一致现象,关键是容易看懂。建议采用表格形式说明。
		注:未申明的变量以其在符号出现处的具体说明为准。
	\end{tabular}
\end{center}


