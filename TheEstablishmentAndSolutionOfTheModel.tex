\newpage	
\section{模型的建立与求解}
\subsection{数据的预处理}


%1. 。。。。。。数据全部缺失,不予考虑。
%2. 对数据测试的特点,如周期等进行分析。
%3. 。。。。。。数据残缺,根据数据挖掘等理论根据。。。。。变化趋势进行补充。
%4. 对数据特点(后面将会用到的特征)进行提取。
%用。。。。。。。软件聚类分析和各个不同问题的需要,采得。。。组采样,每组5-8个采样值。将采样所对应的特征值进行列表或图示。
%根据数据特点,对总体和个体的特点进行比较,以表格或图示方式显示。
\begin{enumerate}[
	label = (\arabic*),
	itemindent = 0pt,
	labelindent = \parindent,
	labelwidth = 2em,
	labelsep = 5pt,
	leftmargin = *]
	\item[$      Step 1$] \quad good morning...
	\item[$      Step 2$] \quad good morning....
\end{enumerate}


\subsection{问题一的模型建立与求解}
\subsubsection{***模型的建立}


%模型建立的内容要点如下:
%模型的主要类别:
%几种常见的建模目的:
%建模过程常见的几个要点:
%模型的基本要求:
%模型选择要点:
%加分项(能在规定时间内做完后还有足够时间的再考虑加分项):
%1、鼓励创新。在能解决问题的基础上,对经典模型进行改进,欣赏独树一帜、有创新性的模型,但要合理。
%2、对于同一问题使用两个或以上合理模型进行求解。避免出现单纯罗列模型,又不做对比和评价的现象。
%参考话术:我们需要解决的问题是。。。。,题目要求是。。。。,剔除。。。数据后选用何种类型的模型优点进行分析。具体步骤123。。。
\subsubsection{***模型的求解}


%将预处理数据带入上述模型,通过。。。软件得到。。结果。(编程代码详见附件*)。模型求解及结果需要图文并茂,用数据说话  用图展示。具体步骤123。。。
\subsubsection{结果}


%针对于每一个问题的结果综述总结。
\subsection{二级标题}
\subsubsection{三级标题}


\subsubsection{列表环境}
\begin{enumerate}
	\item 
	
	\item 	
\end{enumerate}


\begin{itemize}
	\item[(a)] 
	
	
	\item[(b)] 	
\end{itemize}


\begin{itemize}
	\item 
	\item 
\end{itemize}

\subsubsection{图}	
\begin{figure}[h]%[h]:固定作用
	\centering%置中
	\includegraphics [scale=0.5]{figures/1.png}
	\caption{图名} 
	\label{fig:1}
\end{figure}

TOPS:如何引用看这里:图\ref{fig:1}

\subsubsection{表}

\begin{table}[!htbp]
	\caption{表名}	\centering
	\begin{tabular}{c c}
		\hline \multicolumn{1}{c} { 焊接区域中心温度} & 时间 $(\mathrm{s})$ \\
		\hline $30^{\circ} \mathrm{C}$ & 0 \\
		$150^{\circ} \mathrm{C}$ & $t_{1}$ \\
		$190^{\circ} \mathrm{C}$ & $t_{2}$ \\
		\hline
		\label{tab:1}
	\end{tabular}
\end{table}

TOPS:如何引用看这里:表\ref{tab:1}

\subsubsection{公式}

$\alpha^2+\beta^2=\gamma^2$

$$ \alpha^2+\beta^2=\gamma^2$$

\begin{equation}\nonumber
	\alpha^2+\beta^2=\gamma^2
\end{equation}

\begin{equation}
	\alpha^2+\beta^2=\gamma^2 \label{1}
\end{equation}

\begin{equation}
	a+b=c \label{2}
\end{equation}

TOPS:如何引用看这里:公式\eqref{1}和公式\eqref{2}