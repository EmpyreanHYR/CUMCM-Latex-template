\maketitle%与上面的\title对应

\begin{abstract}%摘要+摘要+摘要+摘要+摘要


	%第一段——问题重述+简要思想:首先简要叙述所给问题的背景和动机,并分别分析每个小问题的特点(以下以三个问题为例)。根据这些特点说出自己的思想:针对于问题1,采用。。。。。。。。的方法解决;针对问题2用。。。。。。。。的方法解决;针对问题3用。。。。。。。。的方法解决。
	
	
	
	%\textbf{}:加粗	
	\textbf{针对问题一},	%第二段——模型建立及求解结果:介绍思想和模型: 对于问题1我们首先建立了。。。。。。。。模型I。首先利用。。。。。。,其次计算了。。。。。。,并借助。。。。。。数学算法和。。。。。。软件得出了。。。。。。结论。
	
	
	
	
	\textbf{针对问题二},%(第3段)	对于问题2我们用。。。。。。。。
	
	
	
	\textbf{针对问题三},%(第4段)	对于问题3我们用。。。。。。。。(模型的建立与求解结果的陈述中,思想、模型、软件和结果必须描述清晰,亮点详细说明需突出。针对不同问题可独立成段也可采用一段式仅用分号“;”分割,摘要只接受文字描述形式,不接受图表等其他方式)
	
	
	
	
	
	%(第5段)	优化结果及总结:在。。。。。。条件下,针对。。。。。。模型进行适当修改与优化,这种条件的改变可能来自你的一种猜想或建议。要注意合理性。此推广模型可以不深入研究,也可以没有具体结果。
	
	%注:字数300~600之间,需控制在一页;摘要中必须将具体方法、模型和所得结果写出来;摘要要求“总分总”,段开头可用“针对问题1,针对问题2,针对问题3..”或者“首先,然后,其次,最后”等词语进行有逻辑的论述。摘要是重中之重,必须严格执行!
	
	
	
	\keywords{关键词1\quad  关键词2\quad   关键词3\quad   关键词4\quad 关键词5}
	%\keywords:关键词;\quad:空格
	%使用到的模型名称、方法名称、特别是亮点一定要在关键字里出现,3~5个较合适,用分号隔开
\end{abstract}
%\section{}-subsection{}-subsubsection{}:标题1-3级
\section{问题重述}
\subsection{问题背景}
%在保持原题主体思想不变下,可以自己组织词句对问题进行描述,主要数据可以直接复制,对所提出的问题部分基本原样复制。篇幅建议不要超过一页。大部分文字提炼自原题。)



\subsection{问题提出}
%结合以上情况,建立数学模型解决以下问题:
\begin{enumerate}
	\item 
	\item 
	\item 
	\item 
	\item
\end{enumerate}